\documentclass{article}
\usepackage[utf8]{inputenc}

\title{Summary of Interests}
\author{richard.n.holt }
\date{January 2019}

\usepackage{natbib}
\usepackage{graphicx}

\begin{document}

\maketitle

\section{Economics and Data Science}
My interest in Economics stems from my interest in government policy and economic policies specifically. Specifically, I am interested in the outcomes and consequences of specific policy decisions. With a focus in big data, I will be better equipped to pursue research revolving these outcomes and consequences. I want to be a much more skilled manipulator of data, and I want to be able to use data-driven arguments in my hypothesis testing.

I believe that this class will give me those necessary skills and prepare my for various job opportunities that will require a background in coding and data manipulation. There is little that an economist can do without addressing the large reserves of data currently available to us, and there are many opportunities to consolidate that data into workable sets.

My idea for the research project in this class is to look at significant past economic regulation landmarks and look at them as indicators for changes in growth over time. Rather than analyze short-term changes to the economy that may or may not be the result of an exceedingly specific policy, I want to take large sets of data over extended periods of time to see if such changes can be attributed to long-term government behavior. This would include policy changes like the Supreme Court's decision in Lochner v. New York, extended decreases in taxes, and raising regulations like the minimum wage.

After graduating, I hope to pursue a PhD in Economics. I would like to use this level of education to do research at some level for government policy. I am not sure whether I would want to work directly for the government or for a think tank of some kind, but I would like to do focused research on economic policy in the United States and regarding the global economy. I would like this research to be more qualitative than predictive, but I want to use my experience in this class to learn how to back it and present it with data.

\section{Equation}

\[a^2 + b^2 = c^2\]


\end{document}
