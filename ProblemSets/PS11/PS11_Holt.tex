\documentclass{article}
\usepackage[utf8]{inputenc}

\title{Data Science Research Paper}
\author{Richard Holt}
\date{April 2019}

\usepackage{natbib}
\usepackage{graphicx}

\begin{document}

\maketitle

\section{Introduction}
The purpose of this research is to assess the relationship between immigration and economic growth. As economies around the world develop more diverse and structured markets, questions about immigration policies arise once again. Economists have long debated the merits of an open border policy with regards to a country's comparative advantage. It is intuitive to think that a correlation would exist between fluid travel and economic growth, but the goal of this research is to determine which variable is leading.

Several sources were referenced with the goal of determining the relationship between immigration and growth. This section will include a summary of these articles and a synthesis of the information within them and how it relates to this research.
\citep{schou:2006}
\citep{dancygier:2012}
\citep{migrants}
\citep{hanson:2017}
\citep{simpson:2013}
\section{Data Compilation}
The data gathered for this analysis needed to provide information about immigration patterns both between states and between countries. 

\section{Variable Controls}
In assessing the effects of immigration, the data had to assess the impacts of a number of confounding variables. The variables that were most important to control for include
seasonality, changes in national policy, and changes in private interests with regards to cost of labor.

Immigration occurs for a number of reasons between countries and between states, and assessing the reason for immigrating is an important part of understanding its relationship to growth.

\section{International Analysis}
The intention of this section is to assess the relationship between net national migration and the rate of national GDP growth. This research assesses that relationship in a number of ways, most simply by looking at the relationship between the two over time. In addition to assessing the significance of the relationship at face value over a time series, this analysis contemplates the statistical relationship between the two. Decomposition of the variables provide some understanding of the different factors underlying the analysis including seasonality, randomness, etc.

\section{Interstate Analysis}
The intention of this section is to assess the relationship between state migration patterns and the rate of that state's GDP growth. This relationship is assessed through the lens of microcosmic instances of the same aggregate relationship. The data is assessed in largely the same ways to test the strength of the correlation.

\section{Correlative/Causal Implications}
The aggregate data is used to train models. The predictive power of those models is used to test the state-level data. These predictive models and other tests are used to interpret the correlative and causal implications of the data.

\section{Conclusion}
The results of the different tests and the general conclusions of the data are summarized here. 

\bibliographystyle{plain}
\bibliography{PS11_Holt}
\end{document}
